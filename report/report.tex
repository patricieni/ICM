\documentclass[a4paper]{article}
\frenchspacing

\def\tup#1{\langle#1\rangle}

%% Language and font encodings
\usepackage[english]{babel}
\usepackage[utf8]{inputenc}
\usepackage[T1]{fontenc}
\usepackage[autostyle]{csquotes}
\usepackage[ocgcolorlinks,pdfusetitle]{hyperref}

% for doi link
\usepackage{doi}

%% Sets page size and margins
\usepackage[a4paper,top=3cm,bottom=2cm,left=3cm,right=3cm,marginparwidth=1.75cm]{geometry}
% * <patric.fulop@gmail.com> 2016-11-23T00:32:01.744Z:
%
% ^.
%% Useful packages
\usepackage{amsmath,amssymb}
\usepackage{txfonts}
\usepackage{graphicx}

\usepackage[colorinlistoftodos]{todonotes}
%\usepackage{biblatex}
\usepackage{natbib}
\usepackage{authblk}
\bibliographystyle{alpha}
\DeclareMathOperator{\E}{\mathbb{E}}
\DeclareMathOperator{\p}{\textbf{p}}
\DeclareMathOperator{\q}{\textbf{q}}
\DeclareMathOperator{\W}{\mathbb{W}}
\DeclareMathOperator{\boldH}{\textbf{H}}
\DeclareMathOperator{\boldV}{\textbf{V}}
\DeclareMathOperator{\boldX}{\textbf{X}}
\DeclareMathOperator{\boldY}{\textbf{Y}}
\DeclareMathOperator{\boldx}{\textbf{x}}
\DeclareMathOperator{\boldy}{\textbf{y}}
\DeclareMathOperator{\boldB}{\textbf{B}}
\DeclareMathOperator{\R}{\mathbb{R}}
\DeclareMathOperator{\boldh}{\textbf{h}}
\DeclareMathOperator{\boldv}{\textbf{v}}
\DeclareMathOperator{\arrow}{\rightsquigarrow}
\DeclareMathOperator{\daggerf}{f^{\dagger}}
\DeclareMathOperator{\daggerT}{T^{\dagger}}
\DeclareMathOperator{\bra}{\langle}
\DeclareMathOperator{\ket}{\rangle}
\DeclareMathOperator{\alphastar}{\alpha^*}
\DeclareMathOperator{\xprime}{x'}
\DeclareMathOperator{\wdata}{\W(\p_d,\p_\theta)}
\DeclareMathOperator{\wempirical}{\W_\gamma(\hat{\p},\p_\theta)}
\DeclareMathOperator{\klempirical}{KL({\hat{\p} || \p_\theta})}
\DeclareMathOperator{\kldata}{KL({\p_d || \p_\theta})}

\newtheorem{lemma}{Lemma}
%\addbibresource{Wasserstein_RBM.bib}
\title{ICM Collaboration Notes}
\author{Patric Fulop \& Alex Agachi}
%{s1043702@sms.ed.ac.uk}
\affil{The University of Edinburgh}

\begin{document}
\maketitle


\section{Introduction}
Identify potential, explain problem, literature review, explain briefly what you're predicting in light of lit review. Then conclude by using different benchmarks. Obviously explain your model at some point. 
\section{Data statistics}
There are two main datasets, one with biological information and one with clinical data. 
We give a brief description of the merged dataset before preprocessing:
%
\begin{itemize}
\item \textbf{Key 1:} Patient ID - this is not unique across rows
\item \textbf{Key 2:} Surgery date and clinical surgery date. 
These are sometimes off by one day so we took only surgery dates as being relevant.
\end{itemize}
%
There are a total of \textbf{7825} entries and \textbf{6688} unique patients. Each patient has \textbf{30} relevant attributes. For convenience, the attribute names have been renamed more intuitively, and in English :). \\
\\
Some of the attributes have missing values.
\begin{enumerate}
\item Diagnostic dates are there only for one fifth of the patients,  \textbf{1162}.
\item Date of birth (DoB) is missing for \textbf{1002} patients.
\item Date of death is missing for \textbf{4908} entries, should we assume these are survivors? 
\item Gene data is very sparse, i.e. \textbf{Ch} markers.
\item Gender data has \textbf{332} entries missing.
\end{enumerate}
\subsection{Dealing with Missing data}
Examples. \\
Clearly some patients underwent some tests, while others did not. This is a problem we can deal with in a very robust manner as long as we can assume that the data is missing at random (i.e. the mechanism by which it is missing can be described as random, and does not contain relevant information in itself. For example whether a test was conducted or not for a patient does not say something highly relevant about that patient?s condition/survival expectation in itself.) Even if the data is not missing at random, similar techniques would be applied by default of statistics having invented better ones to date, but it would help to understand better the missing data reasons/mechanisms for our variables, to make sure we describe it properly.

%\subitem IDHmut>IDHwt - 2 classes 
%\subitem Codel >IDHmut> IDHwt (3 classes)
%\subitem IDHmut-Tertmut>IDHmu-Tertwt>IDHwtTertwt>IDHwtTertmut (4 classes)

%MGMTm -binary
%P16 loss binary 
%EGFR amplitude - binary 
 %Loss Chr10 - binary 


\begin{table}[tb]
\vskip 3mm
\begin{center}
\begin{small}
\begin{sc}
\begin{tabular}{lcccr}
\hline
%\abovespace\belowspace
Attribute & Present & Missing & Encoding &Type \\
\hline
%\abovespace  \\
Age at surgery & to see & to see& Age & Numerical discrete  \\%Age at chemo
Gender  & 7493 & 332 & Gender & Binary\\
Histo Grade    & 7825& 0&  Tumor Grade & Categorical (4)      \\
Histo Type    & 7825& 0&  Tumor Type & Categorical   \\
KPS    & ?& ?&  ?  &?    \\
Outcome & 4766 & 3059 & Surgery Type & Categorical (3)\\
Radiotherapy    & 2722& 5103&  Rx Date  & Time $\to$ Ultimately Binary \\
Chemotherapy   & 2950& 4875&  Chemo Date  &Time $\to$ Ultimately Binary    \\
IDH Mutation 1   & 7327& 498&  Gene IDH1 & Categorical (3)   \\
IDH Mutation 2   &7078 & 747&  Gene IDH2   & Categorical (3) \\
Htert C228T & 4336 & 3489 & Gene C228T & Categorical (3)\\ 
Htert C250T & 4333 & 3492 & Gene C250T  & Categorical (3) \\ 
%\belowspace
\hline
\end{tabular}
\end{sc}
\end{small}
\caption{Present and missing variables and their encoding}
\label{tab:sample-table}
\end{center}
\vskip -3mm
\end{table}

\section{Encoding clarifications and target variables}
As previously discussed, in the first phase we are interested in a smaller subset of attributes. The table above indicates some of the variables of interest. Please let us know if we got the right ones and whether we should add more from the dataset. For some of them, some things remain unclear.\\
\begin{enumerate}
\item We aim to add age at surgery as one variable, taking into account surgery date and date of birth. 
\item We do not have any attribute for KPS (performance status score) as far as we know. 
\item The outcome is encoded in the surgery type variable (\textbf{ref:image1}). It is either a type of surgical removal or biopsy. For this variable, does missing data tell us that there was no surgery or that we do not know the outcome?
We can see "aucune" as a type so I would assume that we do not know the outcome. 
\item For radiotherapy and chemotherapy, should we assume that if the patient does not have a date, he did not undergo that treatment? 
\item For IDH mutations (\textbf{ref:image2}), IDH1 and IDH2 seem to predominate there. Are these the two main ones we are interested in? You mentioned IDH wild type which, I assume is the case for non-mutated IDH gene, so I would just say this is the \textbf{Normal} value of IDH1/IDH2 Gene. Is this correct? Furthermore, what does the value "NC" stand for?
\end{enumerate}
%
\textbf{Multiple entries:} On average, patients have more than 1 entry according to how many surgeries they went through. Is that correct and do you have any pointers as to how to treat patients with multiple surgeries? Should we aggregate them or treat them independently? \\
\\ 
\textbf{Target Variable:}
We figured several ways of building such a target variable from the dataset, but we much prefer confirming with you this crucial point.
Can we assume that all patients where there is no death date specified, are still alive (as opposed to them not being alive anymore, but this record being missing)?
Potential target: Death or no death. Diagnostic - death time. Surgery - death time. 
Qualitatively explain what is the target variable, i.e. a mix of surgery date, cancer detection date and death. The dates are not cool variables to model. Change this as we progress. Also age is an important factor. \\ 
\begin{enumerate}
\item Target variables could be time between surgery date and death or just plain and simple death (binary var)
\item What variables should we use to make the most out of it? Ask Marc once we have a few ideas. 
\end{enumerate}
%
\textbf{Equivalence between classes:}
In terms of genetic tests, is there any equivalence between the following coding schemes for answers:\\
%Normal, Amplified
%Normal, increased, decreased, partial, 
% Add images here from notebook, easiest way. 
%Can we treat normal as carrying the same meaning across these schemes? 
\\

\textbf{talk about the feature standardization here and how to deal with class imbalances}


\section{Literature Review}
Find benchmarks and comparison situations for our pipeline and results, i.e. 'Cancer survivability'.

\section{Miscellaneous}
Add whatever crosses your mind

\end{document}